\chapter{Welches Vorwissen hatte ich ? }

Im Vorfeld auf das Modul \dq Soft Skills \dq hatte ich noch kein Vorwissen zu diesem Thema. Vorträge, Gesprächsrunden und Ähnliches waren mir bis dahin sehr fremd. Um meine Meinung darüber zu verbalisieren, möchte ich gern ein Zitat von Madame Pompadour anfügen.

\begin{quote}
	\dq{Man kann mit jedem Menschen reden. Die Kunst besteht darin, es zu vermeiden.} \dq \\
     Madame Pompadour
\end{quote}

Dennoch bildet der mündliche Austausch von Meinung über Sprache und Rede einen wichtigen Grundstein in der heutigen Gesellschaft.  \\

Doch was sind nun Soft Skills? 
Soft Skills bezeichnen nicht eine abgeschlossene Menge von Fähigkeiten, sondern viel mehr einen Pool an verschiedenen Skills. Diese definieren sich über persönliche Eigenschaften (Gelassenheit, Geduld, Freundlichkeit), persönliche Werte (Fairness, Respekt), sowie über soziale Kompetenzen (Teamfähigkeit, Kommunikationsfähigkeit)  ~\cite{SoSk} . \\

Dies war im Grunde das Einzige, was ich über jenes Thema wusste und somit die Grundlage meiner anfänglichen Bedenken. Dies änderte sich jedoch rapide.
In dem Modul sollten zwar hauptsächlich die zwischenmenschlichen Kommunikationen eine Rolle spielen, aber auch andere persönliche Skills.\\

Der große Unterschied zu anderen Modulen war die Herangehensweise. Normal lernt man das Wissen in Vorlesungen und wendet dieses in Übungen an. In diesem Fall jedoch konnten wir uns in Gruppen zusammensetzen und Aufgaben als Video lösen.