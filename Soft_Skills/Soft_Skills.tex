\documentclass
[   twoside=false,     % Einseitiger oder zweiseitiger Druck?
    fontsize=12pt,     % Bezug: 12-Punkt Schriftgröße
    DIV=15,            % Randaufteilung, siehe Dokumentation "KOMA"-Script
    BCOR=17mm,         % Bindekorrektur: Innen 17mm Platz lassen. Copyshop-getestet.
%    headsepline,
    headsepline,  % Unter Kopfzeile Trennlinie (aus: headnosepline)
    footsepline,  % Über Fußzeile Trennlinie (aus: footnosepline)
    open=right,        % Neue Kapitel im zweiseitigen Druck rechts beginnen lassen
    paper=a4,          % Seitenformat A4
    abstract=true,     % Abstract einbinden
    listof=totoc,      % Div. Verzeichnisse ins Inhaltsverzeichnis aufnehmen
    bibliography=totoc,% Literaturverzeichnis ins Inhaltsverzeichnis aufnehmen
    titlepage,         % Titelseite aktivieren
    headinclude=true,  % Seiten-Head in die Satzspiegelberechnung mit einbeziehen
    footinclude=false, % Seiten-Foot nicht in die Satzspiegelberechnung mit einbeziehen
    numbers=noenddot   % Gliederungsnummern ohne abschließenden Punkt darstellen
]   {scrreprt}         % Dokumentenstil: "Report" aus dem KOMA-Skript-Paket
\usepackage{float}
\usepackage{play}
\usepackage[active]{srcltx}
\usepackage[table,xcdraw]{xcolor}
\usepackage{longtable}
\usepackage{subfigure}  	
%\usepackage[activate=normal]{pdfcprot} % Optischer Randausgleich -> pdflatex!
\usepackage{ifthen}
\usepackage[ngerman]{babel}   % Neue Deutsche Rechtschreibung
%\usepackage[latin1]{inputenc} % Zeichencodierung nach ISO-8859-1
\usepackage[utf8]{inputenc}   %	Zeichencodierung nach UTF-8 (Unicode)
\usepackage[T1]{fontenc}
%\usepackage{ae} % obsolet und durch lmodern ersetzt
\usepackage{lmodern}
\usepackage[T1]{url}
\usepackage[final]{graphicx}
\RequirePackage{scrlfile}
\ReplacePackage{scrpage2}{scrlayer-scrpage}
% old: \usepackage[automark]{scrpage2}
\usepackage[automark]{scrlayer-scrpage}
\usepackage{setspace}
\usepackage{cite}
%\usepackage[first,light]{draftcopy} % Für Probedruck
\usepackage[plainpages=false,pdfpagelabels,hypertexnames=false]{hyperref}

% Tiefe der Kapitelnummerierung beeinflussen
\setcounter{secnumdepth}{3} % Tiefe der Nummerierung
\setcounter{tocdepth}{3}    % Tiefe des Inhaltsverzeichnisses

% Hier in die zweite geschweifte Klammer jeweils
% die persönlichen Daten und das Thema der Arbeit eintragen:
\newcommand{\artderausarbeitung}{Reflexionsbericht zum Modul Soft Skills }
\newcommand{\namedesautors}{Jonathan Skopp}
\newcommand{\themaderarbeit}{Kommunikative Kompetenzen im Studium Generale}

% PDF Metadaten definieren
\hypersetup{
   pdftitle={\themaderarbeit},
   pdfsubject={\artderausarbeitung},
   pdfauthor={\namedesautors},
   pdfkeywords={\artderausarbeitung; TU-Ilmenau;Reflexionsbericht zum Modul Soft Skills ;}}


% Abkürzungsverzeichnis beeinflussen. Hier nichts ändern!
\usepackage[intoc]{nomencl}
  \AtBeginDocument{\setlength{\nomlabelwidth}{.25\columnwidth}}
  \let\abbrev\nomenclature
  \renewcommand{\nomname}{Abkürzungsverzeichnis}
  \renewcommand{\nomlabel}[1]{#1 \dotfill}
  \setlength{\nomitemsep}{-\parsep}
  \makenomenclature

\usepackage[normalem]{ulem}
  \newcommand{\markup}[1]{\textbf{#1}}

% Seitenlayout festlegen. Hier nichts ändern!
\pagestyle{scrplain}
\ihead[]{\headmark}
\ohead[]{\pagemark}
\chead[]{}
\ifoot[]{}
\ofoot[]{\scriptsize \artderausarbeitung\ \namedesautors}
\cfoot[]{}
\renewcommand{\titlepagestyle}{scrheadings}
\renewcommand{\partpagestyle}{scrheadings}
\renewcommand{\chapterpagestyle}{scrheadings}
\renewcommand{\indexpagestyle}{scrheadings}

% Abschnittsweise Nummerierung anstatt fortlaufend. Hier nichts ändern!
\makeatletter
\@addtoreset{equation}{chapter}
\@addtoreset{figure}{chapter}
\@addtoreset{table}{chapter}
\renewcommand\theequation{\thechapter.\@arabic\c@equation}
\renewcommand\thefigure{\thechapter.\@arabic\c@figure}
\renewcommand\thetable{\thechapter.\@arabic\c@table}
\makeatother

% Quelltextrahmen, klein. Hier nichts ändern!
\newsavebox{\inhaltkl}
\def\rahmenkl{\sbox{\inhaltkl}\bgroup\small\renewcommand{\baselinestretch}{1}\vbox\bgroup\hsize\textwidth}
\def\endrahmenkl{\par\vskip-\lastskip\egroup\egroup\fboxsep3mm%
\framebox[\textwidth][l]{\usebox{\inhaltkl}}}

% Quelltextrahmen, normale Groesse. Hier nichts ändern!
\newsavebox{\inhalt}
\def\rahmen{\sbox{\inhalt}\bgroup\renewcommand{\baselinestretch}{1}\vbox\bgroup\hsize\textwidth}
\def\endrahmen{\par\vskip-\lastskip\egroup\egroup\fboxsep3mm%
\framebox[\textwidth][l]{\usebox{\inhalt}}}

% Trennvorschläge für falsch getrennte Wörter.
% Wird häufig bei eingedeutschen Wörtern benötigt, da LaTeX hierbei
% gerne falsch trennt. Alternativ kann auch im Fliesstext ein
% Trennvorschlag per "\-" hinterlegt werden, bspw.:
% Die Hard\-ware besteht aus A und B.
\hyphenation{
Hard-ware
}

% Sonstige Befehlsdefinitionen hier ablegen.
\newcommand{\entspricht}{\stackrel{\wedge}{=}}

% Tabellenspaltendefinitionen mit fester Breite --> somit Zeilenumbruch innerhalb einer Zelle möglich
% aus http://www.torsten-schuetze.de/tex/tabsatz-2004.pdf
\usepackage{array, booktabs}
\newcolumntype{f}{>{$}l<{$}}
\newcolumntype{n}{>{\raggedright}l}
\newcolumntype{N}{>{\scriptsize}l}
\newcolumntype{v}[1]{>{\raggedright\hspace{0pt}}m{#1}}
\newcolumntype{V}[1]{>{\scriptsize\raggedright\hspace{0pt}}m{#1}}
\newcolumntype{Z}[1]{>{\raggedright\centering}m{#1}}
\newcolumntype{k}[1]{>{\raggedright}p{#1}}
% ergibt Tabllenspalte fester Breite, linksbündig
% Umbruch innerhalb der Zelle mit \\, neue Tabellezeile mit \tabularnewline
% \addlinespace für Gruppentrennung (aus \texttt{booktabs.sty})


\begin{document}
\onehalfspacing
%% ++++++++++++++++++++++++++++++++++++++++++++++++++++++++++++
%% Hauptdatei, Wurzel des Dokuments
%% ++++++++++++++++++++++++++++++++++++++++++++++++++++++++++++

\begin{titlepage}
\centering
\includegraphics[scale=0.2]{TU-Ilmenau.png}\\[3ex]
{\Large \textsc{Technische Universität Ilmenau}}\\[3ex]
{\Large \Fakultat}\\[3ex]
\vfill
{\Large \textbf{\artderausarbeitung}}\\[4ex]
{\large \textbf{\themaderarbeit}}\\[5ex]
\vfill
\begin{tabular}{rl}
\hline\\
vorgelegt von:          & \quad \namedesautors\\[1,5ex]
eingereicht am:         & \quad \today \\[1,5ex]
Matrikel: 				& \quad \Matrikel \\[1,5ex]
Studiengang:            & \quad \Studienfach\\[1,5ex]
Fakultät:               & \quad \Fakultat \\[1,5ex]
Anfertigung im Fachgebiet:
                        & \quad \Fachgebiet \\[1,5ex]
Verantwortlicher Dozent:
                        & \quad \Dozent \\[1,5ex]
\end{tabular}
\vfill
\end{titlepage}










% Inhaltsverzeichnis
\cleardoublepage % Seitenumbruch erzwingen vor Änderung des Nummerierungsstils
\pagenumbering{roman} % Nummerierung der Seiten ab hier: i, ii, iii, iv...
\pagestyle{scrheadings} % Ab hier mit Kopf- und Fusszeile
\tableofcontents

% Die einzelnen Kapitel
\cleardoublepage % Seitenumbruch erzwingen vor Änderung des Nummerierungsstils
\pagenumbering{arabic} % Nummerierung der Seiten ab hier: 1, 2, 3, 4...

%% Content
\chapter{Welches Vorwissen hatte ich ? }

Im Vorfeld auf das Modul \dq Soft Skills \dq hatte ich noch kein Vorwissen zu diesem Thema. Vorträge, Gesprächsrunden und Ähnliches waren mir bis dahin sehr fremd. Um meine Meinung darüber zu verbalisieren, möchte ich gern ein Zitat von Madame Pompadour anfügen.

\begin{quote}
	\dq{Man kann mit jedem Menschen reden. Die Kunst besteht darin, es zu vermeiden.} \dq \\
     Madame Pompadour
\end{quote}

Dennoch bildet der mündliche Austausch von Meinung über Sprache und Rede einen wichtigen Grundstein in der heutigen Gesellschaft.  \\

Doch was sind nun Soft Skills? 
Soft Skills bezeichnen nicht eine abgeschlossene Menge von Fähigkeiten, sondern viel mehr einen Pool an verschiedenen Skills. Diese definieren sich über persönliche Eigenschaften (Gelassenheit, Geduld, Freundlichkeit), persönliche Werte (Fairness, Respekt), sowie über soziale Kompetenzen (Teamfähigkeit, Kommunikationsfähigkeit)  ~\cite{SoSk} . \\

Dies war im Grunde das Einzige, was ich über jenes Thema wusste und somit die Grundlage meiner anfänglichen Bedenken. Dies änderte sich jedoch rapide.
In dem Modul sollten zwar hauptsächlich die zwischenmenschlichen Kommunikationen eine Rolle spielen, aber auch andere persönliche Skills.\\

Der große Unterschied zu anderen Modulen war die Herangehensweise. Normal lernt man das Wissen in Vorlesungen und wendet dieses in Übungen an. In diesem Fall jedoch konnten wir uns in Gruppen zusammensetzen und Aufgaben als Video lösen.
\chapter{Was habe ich gelernt ? }

Der große Unterschied zu anderen Modulen ist, dass hier praktisches Wissen vermittelt wird. In der Mathematik oder Informatik  bekommt man theoretisches Wisse ohne Bezug auf eine Anwendung gelehrt. \\


 Als Beispiel kann man hier sehr gut die Zwischenmenschliche Kommunikation anführen. In den Videos und dem Skript wurden von Herrn Balkow Verhaltensmuster dargelegt, welche für das gesamte Leben nützlich sind. So erlernte ich dort verschiedenen Varianten kennen, welche eine  gelungene Kommunikation herbeiführt. Die folgende Auflistung stammt aus dem Skript ~\cite{Skript}. Um diese hier einmal zu nennen:
\begin{itemize}
	\item Beobachten
	\item Zuhören
	\item Wahrnehmen
	\item Reflektieren
	\item Lernen
\end{itemize} 

 
Das mag an einigen Stellen jetzt durchaus banal und einfach klingen, jedoch gibt es selbst in der heutigen Zeit viele Probleme, welche man mit einer einfachen Unterhaltung hätte lösen können. Das Schöne ist, das es mir möglich war, mich privat weiter mit diesem Thema auseinander zu setzten. So konnte ich mich mit ~\cite{Schlüssel} und ~\cite{fRede} sehr gut weiterbilden.

\chapter{Was habe ich mitgenommen ?}

Interessanterweise habe ich viel Informatives mitgenommen. Durch die Vorlesungen konnte man sehr viel theoretisches Wissen aufnehmen, welche wir durch praktische Übungen reflektiert haben. Alle Teilnehmer waren also \dq gezwungen \dq , die gestellte Aufgabe zu lösen. Obwohl es sehr anstrengend war, hat es viel Spaß gemacht. 




\section{Rückblick}
\subsection{Gute Aspekte}
Mir hat das Modul viel Freude bereitet. da es mal etwas ganz anderes war. Die schnelle Digitalisierung, welche Corona erfordert hat, wurde sehr gut umgesetzt. 


\subsection{Schlechte Aspekte}
Ich fand keine negativen Objekte in diesem Modul, würde jedoch die Verlinkungen etwas anpassen. So muss man sich in Moodle anmelden um die Prezi zu suchen damit wir dort die Links finden können. Eine Idee wäre es, diese über verschiedene Chapter im Moodle direkt zu publizieren. Es wäre auch schön, wenn man die Prezi hätte herunterladen können.


%% End Content
\appendix
\part*{Anhang}

% Literaturverzeichnis einbinden

\cleardoublepage
\ihead[]{Literaturverzeichnis}
\bibliographystyle{alphadin}
\bibliography{literatur} % "literatur.bib" ist hier die einzige Literaturdatenbank.

%\bibliography{literatur_buecher,literatur_weblinks}

\hspace{100cm}

% Abschlusserklärung
\chapter*{Erklärung}
\addcontentsline{toc}{chapter}{Erklärung}
\ihead[]{Erklärung}

Die vorliegende Arbeit habe ich selbstständig ohne Benutzung anderer als der
angegebenen Quellen angefertigt. Alle Stellen, die wörtlich oder sinngemäß
aus veröffentlichten Quellen entnommen wurden, sind als solche
kenntlich gemacht. Die Arbeit ist in gleicher oder ähnlicher Form oder
auszugsweise im Rahmen einer oder anderer Prüfungen noch nicht vorgelegt
worden.
\\[2cm]
Ilmenau, den 26.\,07.\,2020 \hfill \namedesautors

% Unterschrift
%\includegraphics[width=0.25\textwidth]{Unterschrift.png}
\end{document}