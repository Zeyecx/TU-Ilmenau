
\newpage
\chapter{Dystopie}


\section{Warum sind wir so neugierig nach Utopien?}
Wir sind schon immer daran interessiert wie es in unserer Zukunft aussehen wird. Keiner kann das mit Sicherheit sagen. \\
Verschiedene aktuelle Probleme und Sorgen bringen uns dazu, in der Zukunft danach zu suchen. Schon ganz kindliche Fragen wie \dq Können Autos bald fliegen? \dq könnten der Ursprung einer Dystopie sein. Jeder Gedanke hat die Berechtigung eine Utopie bzw. Dystopie zu werden. Ob nun Eto oder Dysto, dass bleibt dem Erfinder selber überlassen. Fliegende Autos oder Redende Tiere. All das kann eine Zukunfstvision sein. Denn niemand vermag es eindeutig zu beweisen, das es so etwas nicht geben wird. Das es heute noch keine fliegenden Autos gibt, bedarf keines Beweises. Dennoch darf und soll man darüber nachdenken und sich Gedanken machen. ~\cite{dysto}


\section{Warum ist Nier nun eine Dystopie?}
Nier spielt in einer postapokalyptischen Welt, in der es Humanoide gibt. Das eine solche Welt nicht als Wunschvorstellung gilt, sehe ich hier als normal an. Folglich handelt es sich um eine Dystopie. Da das Werk im Jahr 11549 spielt kann man gut erkennen, dass es sich um eine Ferne Zukunft handelt.


\section{Ist so eine Dystopie realistisch ?}
Das wir zur Spielzeit in einer solchen Welt leben ist durchaus denkbar. Durch extreme Misshandlung der Natur kommt es schon heute zu Überflutungen und einer starken Ausbreitung der Wüste. Zum Beispiel ist in den letzten Jahren der Aralsee ausgetrocknet. \\
Jedoch ist zu bezweifeln, dass die Menschheit in naher Zukunft humanoide Roboter entwickeln kann. Auch wenn dies zu wünschen wäre, liegt dieser Wunsch noch in ziemlich weiter Ferne. Der Punkt das wir Roboter brauchen, welche unseren Planeten von gefährlichen Robotern säubern, ist auch gar nicht so weit hergeholt. Produkte wie Siri, Cortana oder ähnliche heute schon das menschliche Verhalten zu imitieren und sich so wie Menschen zu verhalten. Es gibt schon Pflegeroboter wie Paro. Warum sollen wir nicht bald auch Roboter haben, die selber unsere Wohnung aufräumen, putzen oder ähnliches? Doch wie werden sich diese verhalten? Werden sich sich unterwerfen ? Das sind Fragen die erst die Zeit klären kann ? 


\section{Werden Maschinen jemals Gefühle empfinden können?}
Ja. Maschinen werden in der Lage sein, eigene Gefühle zu empfinden. Das wird nicht heute oder morgen passieren, sondern in ferner Zukunft. Dann wäre ein so ähnliches Gespräch wie im Anhang A sehr gut vorstellbar. Dieses Gespräch (hier im original Abgedruckt) stammt aus jenem Werk. Humanoide Androiden werden eigene Gefühle entwickeln, das steht außer Frage. Wie diese sich äußern und welche Form sie annehmen werden, bleibt eine Utopie.

